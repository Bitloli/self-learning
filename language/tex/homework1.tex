\documentclass[fontset=none]{article} %取消CTeX的默认字体设置

\usepackage[UTF8]{ctex}
\usepackage{fancyhdr}
\usepackage{extramarks}
\usepackage{amsmath}
\usepackage{amsthm}
\usepackage{tikz}
\usepackage[plain]{algorithm}
\usepackage{algpseudocode}
\usepackage{lmodern}

% 处理
\usepackage[english]{babel}
\usepackage{hyperref}
\hypersetup{
    colorlinks=true,
    linkcolor=blue,
    filecolor=magenta,      
    urlcolor=cyan,
}

\newtheorem{theorem}{Theorem}

%
% Basic Document Settings
%
\setCJKmainfont{SourceHanSansCN-Light}



\topmargin=-0.45in
\evensidemargin=0in
\oddsidemargin=0in
\textwidth=6.5in
\textheight=9.0in
\headsep=0.25in

\linespread{1.1}

\pagestyle{fancy}
\lhead{\hmwkAuthorName}
\rhead{\firstxmark}
\lfoot{\lastxmark}
\cfoot{\thepage}

\renewcommand\headrulewidth{0.4pt}
\renewcommand\footrulewidth{0.4pt}

\setlength\parindent{0pt}

%
% Create Problem Sections
%

\newcommand{\enterProblemHeader}[1]{
    \nobreak\extramarks{}{Problem \arabic{#1} continued on next page\ldots}\nobreak{}
    \nobreak\extramarks{Problem \arabic{#1} (continued)}{Problem \arabic{#1} continued on next page\ldots}\nobreak{}
}

\newcommand{\exitProblemHeader}[1]{
    \nobreak\extramarks{Problem \arabic{#1} (continued)}{Problem \arabic{#1} continued on next page\ldots}\nobreak{}
    \stepcounter{#1}
    \nobreak\extramarks{Problem \arabic{#1}}{}\nobreak{}
}

\setcounter{secnumdepth}{0}
\newcounter{partCounter}
\newcounter{homeworkProblemCounter}
\setcounter{homeworkProblemCounter}{1}
\nobreak\extramarks{Problem \arabic{homeworkProblemCounter}}{}\nobreak{}

%
% Homework Problem Environment
%
% This environment takes an optional argument. When given, it will adjust the
% problem counter. This is useful for when the problems given for your
% assignment aren't sequential. See the last 3 problems of this template for an
% example.
%
\newenvironment{homeworkProblem}[1][-1]{
    \ifnum#1>0
        \setcounter{homeworkProblemCounter}{#1}
    \fi
    \section{Exercise \arabic{homeworkProblemCounter}}
    \setcounter{partCounter}{1}
    \enterProblemHeader{homeworkProblemCounter}
}{
    \exitProblemHeader{homeworkProblemCounter}
}

%
% Homework Details
%   - Title
%   - Due date
%   - Class
%   - Section/Time
%   - Instructor
%   - Author
%

\newcommand{\hmwkTitle}{Homework One}
\newcommand{\hmwkDueDate}{2020.5.11}
\newcommand{\hmwkClass}{高级算法}
\newcommand{\hmwkClassInstructor}{Professor xiaoming sun}
\newcommand{\hmwkAuthorName}{\textbf{xueshi hu}}

%
% Title Page
%

\title{
    \vspace{2in}
    \textmd{\textbf{\hmwkClass:\ \hmwkTitle}}\\
    \vspace{3in}
}

\author{\hmwkAuthorName}
\date{}

\renewcommand{\part}[1]{\textbf{\large Part \Alph{partCounter}}\stepcounter{partCounter}\\}

%
% Various Helper Commands
%

% Useful for algorithms
\newcommand{\alg}[1]{\textsc{\bfseries \footnotesize #1}}

% For derivatives
\newcommand{\deriv}[1]{\frac{\mathrm{d}}{\mathrm{d}x} (#1)}

% For partial derivatives
\newcommand{\pderiv}[2]{\frac{\partial}{\partial #1} (#2)}

% Integral dx
\newcommand{\dx}{\mathrm{d}x}

% Alias for the Solution section header
\newcommand{\solution}{\textbf{\large Solution}}

% Probability commands: Expectation, Variance, Covariance, Bias
\newcommand{\E}{\mathrm{E}}
\newcommand{\Var}{\mathrm{Var}}
\newcommand{\Cov}{\mathrm{Cov}}
\newcommand{\Bias}{\mathrm{Bias}}

\begin{document}
\maketitle
\pagebreak
\begin{homeworkProblem}
1. 左侧证明: 

由
\[
  n > m > 0 => \frac{n}{m} < \frac{n - 1}{m - 1}
\]
可得
\[
  \binom{n}{m} = \frac{n!}{m!(n-m)!} = \frac{n(n-1)..(n-m+1)}{m(m-1)..1} > \frac{n}{m}\frac{n}{m}...\frac{n}{m} = (\frac{n}{m})^m
\]
左侧证明结束

2. 右侧证明:
已知$e^x$的级数展开
\[
  e^x = 1 + x + \frac{x^2}{2!} + ...
\]
则:

\[
  (\frac{ne}{m})^m = (\frac{n}{m})^m * e^m > (\frac{n}{m})^m * \frac{m^m}{m!} = \frac{n^m}{m!} > \frac{n(n-1)..(n-m+1)}{m!}
\]
右侧证明结束。

综上,该不等式成立。

\end{homeworkProblem}

\begin{homeworkProblem}
第$i$次抛掷得到正面的概率为:
\[
  Pr(X=i) = (1 -p)^{i-1}*p
\]
所以期望为:
\[
  E(X) = \sum_{i=1,2,3..}{(1 -p)^{i-1}*p*i} = p * \sum_{i=1,2,3..}{(1 -p)^{i-1} * i} = \frac{1}{p}
\]
\end{homeworkProblem}

\begin{homeworkProblem}
此处证明Chernoff's Bound的第二部分。

\[
  Pr(X < (1 - \delta)\mu) \leq [\frac{e^{-\delta}}{(1-\delta)^{1-\delta}}]^\mu
\]
取$\lambda=\ln{(1-\delta)}$,由于$f(x)=e^{\ln{(1-\delta)}x}$的单调递减函数,所以:
\[
  Pr(X < (1 - \delta)\mu) = Pr(e^{\lambda{X}} > e^{(1 - \delta)\mu})
\]
由于Markov不等式:
\[
  Pr(e^{\lambda{X}} > e^{(1 - \delta)\mu}) \leq \frac{\E(e^{\lambda{X})}}{e^{\lambda(1 - \delta)\mu}} = e^{(1 - \delta)\mu} = \frac{\E(e^{\lambda{(X_1+X_2+...+X_n)}})}{e^{\lambda(1 - \delta)\mu}}
\]
因为$X_i$是独立的随机变量,所以:
\[
  \frac{\E(e^{\lambda{(X_1+X_2+...+X_n)}})}{e^{\lambda(1 - \delta)\mu}} = \frac{\E(e^{\lambda{X_1}}) * \E(e^{\lambda{X_2}}) * ... * \E(e^{\lambda{X_n}}) }{e^{\lambda(1 - \delta)\mu}}
\]
由于$X_i$是$0-1$随机变量,所以:
\[
  \E(e^{\lambda{X_i}}) = 1-p_i+p_ie^\lambda
\]
由$e^x>x+1$的放缩,以及条件 $\sum{p_i} = \mu$
\[
  \frac{e^{p_1(e^\lambda - 1)} * e^{p_2(e^\lambda - 1)} *...* e^{p_n(e^\lambda - 1)} }{e^{\lambda(1 - \delta)\mu}} 
  = \frac{e^{(p_1+p_2+...+p_n)(e^\lambda - 1)} }{e^{\lambda(1 - \delta)\mu}}
  = \frac{e^{(e^\lambda - 1)\mu} }{e^{\lambda(1 - \delta)\mu}}
\]
带入$\lambda=\ln{(1-\delta)}$,即为
\[
  Pr(X < (1 - \delta)\mu) \leq [\frac{e^{-\delta}}{(1-\delta)^{1-\delta}}]^\mu
\]
证明结束。

\end{homeworkProblem}

\begin{homeworkProblem}
使用LazySelect算法,对于一个输入$S$和$k$,从其中随机选择出来一个子串$R$,然后在子串中间挑选出来$k$的两个范围$l$和$h$,利用$l$和$h$取出$S$的子区间$P$,然后对于子区间$P$使用$n\log(n)$的算法排序,便可以确定$kth$元素。
下面,说明其中的部分,为什么$kth$元素会高概率落入到$l$和$h$中间?\\
给出一下条件:
\begin{itemize}
  \item 从$S$中间一共$n$个元素,即$S$的大小为$n$
  \item 从$S$中间挑选出来$n^{\frac{3}{4}}$个元素,即$S$的大小为$n^{\frac{3}{4}}$
  \item $l$是$R$中间的$max\{kn^{-\frac{1}{4}} - \sqrt{n}, 1\}$,而$h$是$min\{kn^{-\frac{1}{4}} + \sqrt{n}, n^{\frac{3}{4}}\}$元素
\end{itemize}

设$X_i=1$表示在$S$中间随机选择一个元素,其比$kth$小,那么$Pr(X_i=1) = \frac{k}{n}$,$Pr(X_i=0) = 1-\frac{k}{n}$,设$X=\sum_{i}^{n^\frac{3}{4}}{X_i}$表示$R$中间大于$kth$元素的数量,因为$X_i$互相独立,并且是$0-1$分布的,$X$符合二项分布,由Chebyshev 不等式:
\[
  Pr(|X-\E(X)| \ge \sqrt{n}) < \frac{Var(X)}{n} = O(n^{-\frac{1}{4}})
\]
而
\[
  Pr(|X-\E(X)| \ge \sqrt{n}) = Pr(|X-kn^{-\frac{1}{4}}| \ge \sqrt{n})
\]
上式表示,在$S$中间,$(kn^{-\frac{1}{4}} + \sqrt{n})$的元素是小于$kth$元素的,也就是元素未能进入到子区间$P$中间的概率为$O(n^{-\frac{1}{4}})$,同理可以说明另一侧。

综上,LazySelect算法可以在$1-O(n^{-\frac{1}{4}})$的概率下获得$\Theta(n)$的复杂度。


\begin{itemize}
  \item \href{https://www.cs.cornell.edu/courses/cs2110/2009su/Lectures/examples/MedianFinding.pdf}{参考1}
  \item \href{https://cs.stackexchange.com/questions/27685/can-someone-explain-lazyselect}{参考2}
\end{itemize}

\end{homeworkProblem}

\begin{homeworkProblem}
根据二分图构造一个矩阵为:
\[
  A(i,j) = 
   \begin{cases} 
      1 & if(i,j)\in \E\:and\:is\:blue \\
      y & if(i,j)\in \E\:and\:is\:red  \\
      0 & otherwise
   \end{cases}
\]
当存在唯一的完美的红蓝匹配的时候,那么等价于:
\[
  det(A) = p(y) = \sum_{i=0}^{n}{a_i{y^i}}
\]
对于多项式$p(y)$的指数$k$的系数不为0的时候,存在红蓝最佳匹配,判断指数$k$的系数存在可以使用拉格朗日插值的方法。
选择$n+1$个不同的数值,$a_0$,$a_1$,,$a_n$,然后以及对应的$p(a_0)$,$p(a_1)$,,$p(a_n)$。

\end{homeworkProblem}
\end{document}
