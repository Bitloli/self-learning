\documentclass[fontset=none]{article} %取消CTeX的默认字体设置

\usepackage[UTF8]{ctex}
\usepackage{fancyhdr}
\usepackage{extramarks}
\usepackage{amsmath}
\usepackage{amsthm}
\usepackage{tikz}
\usepackage[plain]{algorithm}
\usepackage{algpseudocode}
\usepackage{lmodern}

% 处理
\usepackage[english]{babel}

\newtheorem{theorem}{Theorem}

%
% Basic Document Settings
%
\setCJKmainfont{SourceHanSansCN-Light}



\topmargin=-0.45in
\evensidemargin=0in
\oddsidemargin=0in
\textwidth=6.5in
\textheight=9.0in
\headsep=0.25in

\linespread{1.1}

\pagestyle{fancy}
\lhead{\hmwkAuthorName}
\rhead{\firstxmark}
\lfoot{\lastxmark}
\cfoot{\thepage}

\renewcommand\headrulewidth{0.4pt}
\renewcommand\footrulewidth{0.4pt}

\setlength\parindent{0pt}

%
% Create Problem Sections
%

\newcommand{\enterProblemHeader}[1]{
    \nobreak\extramarks{}{Problem \arabic{#1} continued on next page\ldots}\nobreak{}
    \nobreak\extramarks{Problem \arabic{#1} (continued)}{Problem \arabic{#1} continued on next page\ldots}\nobreak{}
}

\newcommand{\exitProblemHeader}[1]{
    \nobreak\extramarks{Problem \arabic{#1} (continued)}{Problem \arabic{#1} continued on next page\ldots}\nobreak{}
    \stepcounter{#1}
    \nobreak\extramarks{Problem \arabic{#1}}{}\nobreak{}
}

\setcounter{secnumdepth}{0}
\newcounter{partCounter}
\newcounter{homeworkProblemCounter}
\setcounter{homeworkProblemCounter}{1}
\nobreak\extramarks{Problem \arabic{homeworkProblemCounter}}{}\nobreak{}

%
% Homework Problem Environment
%
% This environment takes an optional argument. When given, it will adjust the
% problem counter. This is useful for when the problems given for your
% assignment aren't sequential. See the last 3 problems of this template for an
% example.
%
\newenvironment{homeworkProblem}[1][-1]{
    \ifnum#1>0
        \setcounter{homeworkProblemCounter}{#1}
    \fi
    \section{Note \arabic{homeworkProblemCounter}}
    \setcounter{partCounter}{1}
    \enterProblemHeader{homeworkProblemCounter}
}{
    \exitProblemHeader{homeworkProblemCounter}
}

%
% Homework Details
%   - Title
%   - Due date
%   - Class
%   - Section/Time
%   - Instructor
%   - Author
%

\newcommand{\hmwkTitle}{笔记}
\newcommand{\hmwkDueDate}{2020.5.11}
\newcommand{\hmwkClass}{高级算法}
\newcommand{\hmwkClassInstructor}{Professor xiaoming sun}
\newcommand{\hmwkAuthorName}{\textbf{hu bachelar}}

%
% Title Page
%

\title{
    \vspace{2in}
    \textmd{\textbf{\hmwkClass:\ \hmwkTitle}}\\
    \vspace{3in}
}

\author{\hmwkAuthorName}
\date{}

\renewcommand{\part}[1]{\textbf{\large Part \Alph{partCounter}}\stepcounter{partCounter}\\}

%
% Various Helper Commands
%

% Useful for algorithms
\newcommand{\alg}[1]{\textsc{\bfseries \footnotesize #1}}

% For derivatives
\newcommand{\deriv}[1]{\frac{\mathrm{d}}{\mathrm{d}x} (#1)}

% For partial derivatives
\newcommand{\pderiv}[2]{\frac{\partial}{\partial #1} (#2)}

% Integral dx
\newcommand{\dx}{\mathrm{d}x}

% Alias for the Solution section header
\newcommand{\solution}{\textbf{\large Solution}}

% Probability commands: Expectation, Variance, Covariance, Bias
\newcommand{\E}{\mathrm{E}}
\newcommand{\Var}{\mathrm{Var}}
\newcommand{\Cov}{\mathrm{Cov}}
\newcommand{\Bias}{\mathrm{Bias}}

\begin{document}

\maketitle

\pagebreak

\begin{homeworkProblem}

  介绍了几个类型: RP coRP Bpp Zpp,单边错,双边错,不出错的,它们的之间关系。运用 切比雪夫 不等式。

\end{homeworkProblem}

\begin{homeworkProblem}

  Balls \& Bins : 类似于生日悖论

  将Balls = m Bins = n
  当 $m \sim \theta(\sqrt n)$  的时候, $\exists x_i \ge 2 $ 的概率很高了,也可以表述为 $Y = \max\limits_{1 \leq i \leq m}x_i \ge 2$.

\noindent\rule[0.25\baselineskip]{\textwidth}{1pt}

  Load Balancing: 分布式算法,将 request 随机放到一个机器上 ?

  一个特殊情况,$m = n$的时候,那么$Y$的取值是多少 ?
  \begin{itemize}
    \item 平均情况,负载为$1$
    \item 当 $m = n$ ,那么 $Y \sim \theta(\frac{\ln{n}}{\ln\ln{n}})$ W.H.P
    \item Proof :
      转化为两个部分:
      $Pr(Y \le 4 \times \frac{\ln{n}}{\ln\ln{n}})$ 和
      $Pr(Y \ge \frac{1}{4} \times \frac{\ln{n}}{\ln\ln{n}})$
    \item 第一部分的证明,首先设$ t = \frac{\ln{n}}{\ln\ln{n}}$

    首先分析 $Pr(x_i \ge 4 \times t)$ 的含义: 所有4t+1个球在$x_i$中间的概率,那么这些组合个球在其中的概念之和。
  \[
    Pr(x_i \ge 4 \times t) \le \sum\limits_{1 \leq j_1 \le ... \le j_{4t+1} \leq n }^{j_i ... j_{4t+1}} Pr(j_i ... j_{4t+1} \in bin) = \binom{n}{4t + 1}(\frac{1}{n})^{4t+1}
  \]

  存在公式:
  \[
    (\frac{n}{m})^m \leq  \binom{n}{m} = \frac{n!}{m!(n-m)!} \leq (\frac{ne}{m})^m
  \]

  所以:
  \[
    Pr(x_i \ge 4 \times t) \leq (\frac{ne}{4t+1})^{4t+1}(\frac{1}{n})^{4t+1}
    < (\frac{1}{t})^{4t+1}
    = (\frac{\ln{\ln{n}}}{\ln{n}})^{4t+1}
    < (\frac{\sqrt{\ln{n}}}{\ln{n}})^{4t+1}
    = (\frac{1}{\ln{n}})^{\frac{4t+1}{2}}
  \]

  \[
    < (\ln{n})^{-2\frac{\ln{n}}{\ln{\ln{n}}}} = ({e}^{\ln{\ln{n}}})^{-2\frac{\ln{n}}{\ln{\ln{n}}}} = \frac{1}{n^2}
  \]

  上面说明了,对于任意的一个bin的大于,那么对于含有bin的大于 4t+1 的概率是:

  \[
   Pr(\max\limits_{1 \leq i \leq m}x_i \ge 4 \times t) = Pr(x_1 \ge 4 \times t) ... \cup Pr(x_i \ge 4 \times t) ... \cup Pr(x_n \ge 4 \times t) < n \times \frac{1}{n^2} = \frac{1}{n}
  \]

  到此第一部分的证明,现在证明第二个部分:$Pr(Y \ge \frac{1}{4} \times \frac{\ln{n}}{\ln\ln{n}})$

  \[
    Pr(x_i < \frac{1}{4} \times t) = \sum\nolimits_{k \ge \frac{t+1}{4} } Pr(x_i = k) \ge Pr(x_i = \frac{t}{4} + 1) = \binom{n}{\frac{t}{4} + 1}({\frac{1}{n}})^{\frac{t}{4}+1}(1- \frac{1}{n})^{n - \frac{t}{4} - 1}
  \]

  其中:
  \[
    (1- \frac{1}{n})^{n - \frac{t}{4} - 1} \ge (1- \frac{1}{n})^n \sim \frac{1}{e}
  \]
  所以:

  \[
    Pr(x_i < \frac{1}{4} \times t) \ge \frac{1}{e} (\frac{n}{\frac{t}{4} +1 })^{\frac{t}{4} +1}(\frac{1}{n})^{\frac{t}{4} +1} \ge \frac{1}{e}\frac{1}{\ln{n}}\frac{1}{n^{\frac{1}{4}}} \ge \frac{1}{n^\frac{1}{3}}
  \]

  也就是

  \[
     Pr(x_i < \frac{1}{4}
  \]


% 如果不懂,阅读 2:25:44 的截图吧!

\end{itemize}



\end{homeworkProblem}


\end{document}
